\chapter{Recipe scopes}
\label{app:recipe-scopes}
\paragraph{Class scope} The class scope can enable or disable recipes based on the name and/or package of the class itself or based on the name and package of any classes or interfaces it inherits from. 

\paragraph{Method scope} The {method scope} enables recipes based on the name of the method. These scopes are mostly used to enforce guidelines in inherited methods. For example, when creating a servlet in Java \gls{ee}, it is advised to configure some security headers with the \texttt{doGet} and \texttt{doPost} methods inherited from the \texttt{HttpServlet} class. We do this by enforcing a guideline that states that the \texttt{addHeader} method should be called with specific parameters to set the required headers. We then limit the scope of this guideline to only be enabled when the class inherits from \texttt{HttpServlet} and the method name is  \texttt{doGet} or \texttt{doPost}. Using the \gls{yaml} syntax many properties of the method can be used for the scope, such as the number of parameters, the types of parameters, the return type, and any annotations added to the method.

\paragraph{File scope} The {file scope} is used to enable recipes based on project file names. This is mostly used for configuration files. This allows us for example to enforce coding guidelines in the Android manifest file, as its name is always \texttt{AndroidManifest.xml}. This scope is not yet migrated to the \gls{yaml} syntax.

\paragraph{Android context scope} The {Android context scope} was created to raise context awareness in Android projects. In the Android manifest a developer can configure capabilities of components such as activities and broadcast receivers regarding their communication towards the \gls{os}. They can listen to any other application, or only to authorized applications, or only to the own application. The Android context scope allows us to enable recipes based on the configuration of the relevant component, so that we can enforce different recipes for different levels of exposure. We can for example allow communication of sensitive information between classes that are configured as private components, but not between other classes. This scope is not yet migrated to the \gls{yaml} syntax.

\paragraph{Android build property scope} The {Android build property scope} can be used to enable recipes based on the build property of an Android project. Mostly this is used to look at the \texttt{minSdkVersion} property, to determine what versions of Android the application will be compiled to. Specific version of Android have specific vulnerabilities, so recipes need to be disabled based on that build properties. This scope is not yet migrated to the \gls{yaml} syntax.