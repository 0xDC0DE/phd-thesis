\section{Perspectives}
\label{sec:sensei-perspectives}
%\todo[inline]{intro}

\subsection{Improved recipe creation}
\label{sec:improvedrulecreation}
Security professionals report that Sensei is the easiest tool they have used when it comes to creating new recipes.
They attribute this mostly to the preview panels in the recipe editor, rather than the specific \gls{yaml} syntax.
Developers, on the other hand, are not used to creating rules for any tools, so they usually have nothing to compare it to.
This is evident from the usability tests, where developers were more hesitant and more easily overwhelmed by the recipe editor compared to the security professionals.
To reduce this hesitation, in the previous section, a design was proposed that would make the \gls{ui} view the main focus in the recipe editor.
In this design the code view would only be used for copy-pasting recipes.
However, this still requires developers to create recipes for an analysis tool, a task they are not familiar with.

Instead, it might be possible to let developers create recipes simply by writing code.
Currently, Sensei is able to apply a code transformation based on instructions from a recipe.
In the future, it might be possible to do the opposite, and generate a Sensei recipe from a code transformation.
If this technology exists, the recipe editor can simply show two code panels side by side.
The left panel can be a static view of the current state of the code, while the right panel allows the developer to make (small) changes to the code.
A Sensei recipe can then be created from the code changes that can optionally be adjusted in the next step.

This technology would also enable automatic recipe creation methods, such as generating a recipe from a code patch in the code repository.
It might even be possible to dynamically suggest recipes while observing the developer during their normal workflow.
Previous research has been performed to identify \gls{api} rules for cryptography from code changes~\cite{paletov2018inferring}.
Efforts have also been made to automatically generate patches from code repositories and their histories, using different algorithms~\cite{weimer2009automatically}, including those learned from human-written patches~\cite{kim2013automatic} or correct code~\cite{long2016automatic}.
While this research tries to automatically patch bugs, the approaches can also be used to create recipes to apply the discovered patches more broadly and to do so during the writing of code rather than afterwards.
With some user interaction, such a tool might also be able to generate recipes (without fixes) from the output of traditional security tools.

This technology is part of ongoing research funded by a \gls{vlaio} \gls{oeno} project as of 2019.

\subsection{Adapting feedback to the skill level}
An important concept during the design and evaluation of the Sensei \gls{ide} plugin, is the \gls{efp} rate.
When many markings exist in the code that the developer does not intend to fix, i.e. when there is a high \gls{efp} rate, this might cause developers to be overwhelmed and ignore feedback from Sensei altogether.
To explain this concept, the example of \gls{os} command injection was used.
A simple and easy to understand recipe to avoid \gls{os} command injection, is to simply avoid all uses of \gls{os} commands.
A more experience developer, however, will understand how \gls{os} commands can be used securely, for example to launch a different software application through a hard-coded command.
This recipe will lead to an \gls{efp} for an experienced developer, but might be more easily understood by a developer with no security skills than a more advanced recipe.

In other cases, recipes are created that detect (presumably) deliberate insecure configurations.
Take the example of cookies, where it is generally recommended to configure them as HttpOnly.
This prevents the cookies from being used in client-side scripts, and hence avoids some of the most common \gls{xss} attacks.
However, in some legitimate cases the developer might need to use a cookie in a client-side script, and to configure the cookie as such.
Of course, they have to take the security implication of this configuration into consideration.
For example, they will have to ensure that this is not used for security-sensitive cookies, such as session cookies.
A recipe that detects insecure configurations like this, will lead to \glspl{efp} for developers who need the features that are blocked by these configurations.

Both the example of the \gls{os} command and cookie configuration, lead to \glspl{efp} for security experts, but are still important recipes to enforce for a novice developer.
Fortunately, through integration with the \gls{scw} portal, a user ability estimate is available, such that the feedback for recipes like this can be adapted to the skill level of the developer.
For developers with a low ability level, this recipe can be shown as an error, while the more experienced developer can be shown a warning or information level marking.
The descriptions can also be adapted for each skill level.
The less experienced developer can be shown the simple and easy to understand guideline, to use the most secure configuration.
To a more skill full developer it will be less overwhelming to explain the security implications of the configuration, and how to mitigate them in other parts of the code.
\todo[inline]{read refs and revisit}
Adapting \glspl{ui} to the ability level of a user like this, has been researched before with varying degrees of success~\cite{johnson2015bespoke, cockburn2014supporting}.