\section{Perspectives}
\label{sec:sensei-perspectives}
%\todo[inline]{intro}

\subsection{Improved recipe creation}
\label{sec:improvedrulecreation}
Security professionals report that Sensei is the easiest tool they have used when it comes to creating new recipes.
They attribute this mostly to the preview panels in the recipe editor, rather than the specific \gls{yaml} syntax.
Developers, on the other hand, are not used to creating rules for any tools, so they usually have nothing to compare it to.
This is evident from the usability tests, where developers were more hesitant and more easily overwhelmed by the recipe editor compared to the security professionals.
To reduce this hesitation, in the previous section, a design was proposed that would make the \gls{ui} view the main focus in the recipe editor.
In this design the code view would only be used for copy-pasting recipes.
However, this still requires developers to create recipes for an analysis tool, a task they are not familiar with.

Instead, it might be possible to let developers create recipes simply by writing code.
Currently, Sensei is able to apply a code transformation based on instructions from a recipe.
In the future, it might be possible to do the opposite, and generate a Sensei recipe from a code transformation.
If this technology exists, the recipe editor can simply show two code panels side by side.
The left panel can be a static view of the current state of the code, while the right panel allows the developer to make (small) changes to the code.
A Sensei recipe can then be created from the code changes that can optionally be adjusted in the next step.

This technology would also enable automatic recipe creation methods, such as generating a recipe from a code patch in the code repository.
It might even be possible to dynamically suggest recipes while observing the developer during their normal workflow.
Previous research has been performed to identify \gls{api} rules for cryptography from code changes~\cite{paletov2018inferring}.
Efforts have also been made to automatically generate patches from code repositories and their histories, using different algorithms~\cite{weimer2009automatically}, including those learned from human-written patches~\cite{kim2013automatic} or correct code~\cite{long2016automatic}.
While this research tries to automatically patch bugs, the approaches can also be used to create recipes to apply the discovered patches more broadly and to do so during the writing of code rather than afterwards.
With some user interaction, such a tool might also be able to generate recipes (without fixes) from the output of traditional security tools.

This technology is part of ongoing research funded by a \gls{vlaio} \gls{oeno} project as of 2019.

\subsection{Adapting feedback to the skill level}
An important concept during the design and evaluation of the Sensei \gls{ide} plugin, is the \gls{efp} rate.
When many markings exist in the code that the developer does not intend to fix, i.e., when there is a high \gls{efp} rate, this might cause developers to be overwhelmed and ignore feedback from Sensei altogether.
To explain this concept, the example of \gls{os} command injection was used.
A simple and easy to understand recipe to avoid \gls{os} command injection, is to simply avoid all uses of \gls{os} commands.
A more experienced developer, however, will understand how \gls{os} commands can be used securely, for example to launch a different software application through a hard-coded command.
This recipe will lead to an \gls{efp} for an experienced developer, but might be more easily understood by a developer with no security skills than a more advanced recipe.

In other cases, recipes are created that detect (presumably) deliberate insecure configurations.
Take the example of cookies, where it is generally recommended to configure them as HttpOnly.
This prevents the cookies from being used in client-side scripts, and hence avoids some of the most common \gls{xss} attacks.
However, in some legitimate cases the developer might need to use a cookie in a client-side script, and to configure the cookie as such.
Of course, they have to take the security implication of this configuration into consideration.
For example, they will have to ensure that this is not used for security-sensitive cookies, such as session cookies.
A recipe that detects insecure configurations like this, will lead to \glspl{efp} for developers who need the features that are blocked by these configurations.

Both the example of the \gls{os} command and cookie configuration, lead to \glspl{efp} for security experts, but are still important recipes to enforce for a novice developer.
Fortunately, through integration with the \gls{scw} portal, a user ability estimate is available, such that the feedback for recipes like this can be adapted to the skill level of the developer.
For developers with a low ability level, this recipe can be shown as an error, while the more experienced developer can be shown a warning or information level marking.
The descriptions can also be adapted for each skill level.
The less experienced developer can be shown the simple and easy to understand guideline, to use the most secure configuration.
To a more skillfull developer it will be less overwhelming to explain the security implications of the configuration, and how to mitigate them in other parts of the code.
Adapting \glspl{ui} is most often based on the experience of the user with the interface itself rather than their knowledge in a specific field~\cite{johnson2015bespoke, cockburn2014supporting}.
This research indicates that optimizing \gls{ui} design based on novice learning rather than long-term efficiency by experienced users can be counterproductive.
It remains future work to assess whether or not a more optimised \gls{ui} will be required for advanced Sensei users.

\subsection{\hl{Controlled experiment in industry environment}}

The experiments in this part of the book mostly contain second-hand information from industry trials, as well as a controlled experiment with students that focuses on validating some of the features contributing to the usability of Sensei.
In contrast to the first part of this book, no strong empirical evidence is presented that validates the efficacy of a tool like Sensei, or the paved path methodology in general.

In this section, I discuss the design of future experiments that could be conducted to gather stronger evidence for the proposed solution.

\subsubsection{Subjects}
In an industry setting, it is not practical to divide the subjects into a control group and test group of similar skill level.
Instead it might be feasible to compare two teams, and allow the use of Sensei on one of the teams.
Alternatively, the same team could first be observed for a duration of several days or weeks without use of Sensei and then Sensei can be introduced and results can be compared over time.

\subsubsection{Research questions}
In the controlled empirical experiment of Section~\ref{sec:experiment}, the focus was on validating the effectiveness of several Sensei features at helping developers adhere to secure coding guidelines during development.
The results showed that Sensei code markings were addressed often, and were addressed fast. 
Of the guidelines violations introduced by subjects of the test group, 98.4\% were resolved with a mean remediation time of 19.10 s.

However, developers in an industry environment are likely to use Sensei differently to the subjects of this experiment.
Not only because they are more experienced, and are likely more familiar with alternative tools, but also because they work in larger teams and with stricter deadlines.

\paragraph{Feature validation}
In one experiment that can be conducted in the future, the same features can be evaluated in an industry environment.
The \textit{goal} of this experiment would be to observe the impact of the Sensei plugin on the developers in an industry setting.
The \textit{purpose} would be to evaluate the usability and effectiveness of some of the features of Sensei.
The \textit{quality focus} is the ability of the plugin to help developers adhere to the secure coding guidelines that are in place and the impact on their cognitive burden while doing so.

The following research questions can be answered in this experiment:
\begin{itemize}
    \item \textbf{Q1} What is the effective false positive rate in an industry environment?
    \item \textbf{Q2} What is the mean remediation time of markings in an industry environment?
    \item \textbf{Q3} Do developers in an industry environment often use the provided quick-fixes to resolve code markings?
\end{itemize}

This data could be analyzed and compared to the findings of the experiment with students.
This could both evaluate the Sensei features more thoroughly as well as give us insights into the differences between developers in educational institutions and developers in industry environments.
Experiments like this can help researchers and businesses of security tools improve the usability and effectiveness of their tools.
However, in order for organizations to adopt these tools more widely, it would be more impactful to study their effect in a development environment.

\paragraph{Effect on development}
In industry environments, a trade-off exists between fast delivery of new software features and the security of the software.
Improving one at the detriment of the other is not real improvement, and so in this experiment, both these metrics have to be taken into account.
In this experiment, the \textit{goal} is to measure the impact of role-specific security tools for developers on development and delivery of software.
The \textit{purpose} is to evaluate the \gls{roi} for organizations that consider using such a tool during development.
The \textit{quality focus} is the impact of the tool on the delivery speed as well as the security of the delivered code.

The above goal can be achieved by means of an experiment aimed at answering the following questions:
\begin{itemize}
    \item \textbf{Q1} Is the speed of delivery impacted meaningfully when developers start using Sensei?
    \item \textbf{Q2} Is the number of findings by security tools deployed later in the \gls{sdlc} impacted meaningfully when developers start using Sensei?
\end{itemize}

Delivery speed can be measured through several metrics~\cite{software-delivery-performance}.
\begin{itemize}
    \item \textit{Lead time for change} is the time it takes between a customer request and the request being satisfied.
    \item \textit{Deployment frequency} is the frequency with which working software is deployed to production or distributed to the customers
    \item \textit{Change failure rate} is the frequency with which deployments require rollback or other measures to amend failures
    \item \textit{Time to restore service} is the time it takes to restore service after an incident
\end{itemize}

\paragraph{Efficacy in supporting the paved path methodology}
Finally, an experiment could be set up to evaluate if the use of Sensei help development and security teams to adopt practices in line with the paved path methodology.
The \textit{goal} of this experiment is to observe the impact of Sensei on the processes and interactions between security professionals and developers.
The \textit{purpose} is to evaluate if Sensei is effective at supporting the paved path methodology.
The \textit{quality focus} is on the interactions between different team members and the purpose of the Sensei rules that are being developed.

The above goal can be achieved by means of an experiment aimed at answering the following questions:
\begin{itemize}
    \item \textbf{Q1} Do security professionals frequently need to consult developers to create quick-fixes?
    \item \textbf{Q2} Do developers frequently write rules for other purposes than security?
\end{itemize}

