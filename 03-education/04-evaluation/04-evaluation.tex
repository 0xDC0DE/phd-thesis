\chapter{Discussion and perspectives}
\label{ch:its-evaluation}

In the previous chapter, I described the goal and set-up of each experiment, and reported their findings.
In this chapter, I summarize the findings and explain the learned lessons.
I describe how these results can be used to provide better support and training to developers. 

\summarybox{
The \gls{2pl} model has shown that some infamous vulnerability types that are typically considered high priority are relatively easy to find and fix in training.
This is evidence of a gap between knowledge and practice.
The paved path methodology, together with more usable developer tools, could help developers apply their knowledge better.

For use in the \gls{its}, the \gls{knn} baseline algorithm is the most accurate \gls{cf} algorithm.
Baselines have proven to be accurate for the data of the platform, most likely because many users show consistent bias in their ratings.
This is related to the current item selection, that is often consistently right or consistently wrong, based on the ability level of the user.
In literature, model-based algorithms are usually more accurate, but their advantages of dealing with data sparsity, scalability, synonyms, and implicit data are not applicable to the data set of the \gls{scw} training platform.

The results of the \gls{2pl} model and the \gls{its} will be gradually implemented in the training platform.
In the future the \gls{its} will be improved to use data gathered from other developer tools, as described in a published patent by \gls{scw}.
}

\section{Discussion}
\label{sec:its-discussion}

\subsection{Rasch model}
Many efforts exist to rate, rank, organize, and prioritize vulnerabilities into lists and taxonomies, such as the \gls{owasp} Top 10~\cite{wichers2017owasp}, the seven pernicious kingdoms~\cite{tsipenyuk2005seven}, the \gls{cve}~\cite{guo2009ontology,mann1999towards,baker1999development}, and the \gls{cwe}/SANS Top 25 most dangerous software errors~\cite{martin20112011}.
They are often built from the perspective of the security professional, and take into account prevalence in production, detectability by tools, and potential of impact.
These lists and taxonomies can be used as guidelines for security professionals to decide which insecurities should be prioritized.

The results from the Rasch model sometimes contradict these lists.
Injection and \gls{xxe}, for example, are typically rated high on prioritization lists because of their prevalence in production software.
According to our data, however, they are not the most difficult challenges in training.
When the developer is aware such a vulnerability is present, they are able to detect and resolve it with relative ease.
So while popular lists like \gls{owasp} Top 10 can be useful as baselines and guidelines for security teams to set their priorities, but they might not be the right focus from an education perspective.
Based on the results from the Rasch model, developer training should go further than these infamous security problems and focus more on security problems involving larger pieces of code.
We can see this shift in priority in the newest draft of the \gls{owasp} Top 10\footnote{\url{https://owasp.org/Top10/}}.
In the new version, shown in Figure~\ref{fig:newowasptop10}, ``Injection" goes down in priority and``\gls{xxe}" even merges with ``Security misconfiguration".
New categories are introduced such as ``insecure design", proposed as the new number 4, shifting the focus towards security flaws.

\begin{sidewaysfigure}
    \centering
    %
\begin{tikzpicture}[
    scale=0.75
    ]
    
    % SDLC blocks
    \coordinate(a1) at (10.0, 10.0);
    \coordinate(a2) at (10.3, 10.5);
    \coordinate(a3) at (10.0, 11.0);
    \coordinate(a4) at (12.8, 11.0);
    \coordinate(a5) at (13.1, 10.5);
    \coordinate(a6) at (12.8, 10.0);
    \fill[scw-yellow] (a1) -- (a2) -- (a3) -- (a4) -- (a5) -- (a6) -- cycle;
    \node[black] at (11.5,10.5) {\sffamily Plan};
    
    \coordinate(b1) at (13.0, 10.0);
    \coordinate(b2) at (13.3, 10.5);
    \coordinate(b3) at (13.0, 11.0);
    \coordinate(b4) at (15.8, 11.0);
    \coordinate(b5) at (16.1, 10.5);
    \coordinate(b6) at (15.8, 10.0);
    \fill[scw-yellow] (b1) -- (b2) -- (b3) -- (b4) -- (b5) -- (b6) -- cycle;
    \node[black] at (14.5,10.5) {\sffamily Develop};
    
    \coordinate(c1) at (16.0, 10.0);
    \coordinate(c2) at (16.3, 10.5);
    \coordinate(c3) at (16.0, 11.0);
    \coordinate(c4) at (18.8, 11.0);
    \coordinate(c5) at (19.1, 10.5);
    \coordinate(c6) at (18.8, 10.0);
    \fill[scw-yellow] (c1) -- (c2) -- (c3) -- (c4) -- (c5) -- (c6) -- cycle;
    \node[black] at (17.5,10.5) {\sffamily Build};
    
    \coordinate(d1) at (19.0, 10.0);
    \coordinate(d2) at (19.3, 10.5);
    \coordinate(d3) at (19.0, 11.0);
    \coordinate(d4) at (21.8, 11.0);
    \coordinate(d5) at (22.1, 10.5);
    \coordinate(d6) at (21.8, 10.0);
    \fill[scw-yellow] (d1) -- (d2) -- (d3) -- (d4) -- (d5) -- (d6) -- cycle;
    \node[black] at (20.5,10.5) {\sffamily Test};
    
    \coordinate(e1) at (22.0, 10.0);
    \coordinate(e2) at (22.3, 10.5);
    \coordinate(e3) at (22.0, 11.0);
    \coordinate(e4) at (24.8, 11.0);
    \coordinate(e5) at (25.1, 10.5);
    \coordinate(e6) at (24.8, 10.0);
    \fill[scw-yellow] (e1) -- (e2) -- (e3) -- (e4) -- (e5) -- (e6) -- cycle;
    \node[black] at (23.5,10.5) {\sffamily Release};
    
    % arrows -- last to first
    % 5th arrow
        %triangle
    \coordinate(i1) at (24.3, 7.2); 
    \coordinate(i2) at (24.1, 7.0); 
    \coordinate(i3) at (24.3, 6.8); 
    
    \coordinate(i4) at (24.3, 6.9); 
    \coordinate(i5) at (24.8, 6.9); 
        % top
    \coordinate(i6) at (24.8, 9.9); 
    \coordinate(i7) at (24.6, 9.9); 
    
    \coordinate(i8) at (24.6, 7.1); 
    \coordinate(i9) at (24.3, 7.1); 
    
    \node[scw-orange,left] at (24.6,9.2) {\footnotesize Breaches};
    \fill[scw-orange] (i1) -- (i2) -- (i3) -- (i4) -- (i5) -- (i6) -- (i7) -- (i8) -- (i9) -- cycle;
    
    % horizontal
    \coordinate(i10) at (24.15, 7.1); 
    \coordinate(i20) at (24.05, 7.0); 
    \coordinate(i30) at (24.15, 6.9); 
    \coordinate(i40) at (21.85, 6.9); 
    \coordinate(i50) at (21.85, 7.1); 
    \fill[scw-orange] (i10) -- (i20) -- (i30) -- (i40) -- (i50) -- cycle;
    
    
    % 4th arrow
        %triangle
    \coordinate(j1) at (21.3, 7.2); 
    \coordinate(j2) at (21.1, 7.0); 
    \coordinate(j3) at (21.3, 6.8); 
    
    \coordinate(j4) at (21.3, 6.9); 
    \coordinate(j5) at (21.8, 6.9); 
        % top
    \coordinate(j6) at (21.8, 9.9); 
    \coordinate(j7) at (21.6, 9.9); 
    
    \coordinate(j8) at (21.6, 7.1); 
    \coordinate(j9) at (21.3, 7.1); 
    
    \fill[scw-orange] (j1) -- (j2) -- (j3) -- (j4) -- (j5) -- (j6) -- (j7) -- (j8) -- (j9) -- cycle;
    % horizontal
    \node[scw-orange, left] at (21.6,9.2) {\footnotesize Code};
    \node[scw-orange, left] at (21.6,8.7) {\footnotesize analysis};
    \node[scw-orange, left] at (21.6,8) {\footnotesize Penetration};
    \node[scw-orange, left] at (21.6,7.5) {\footnotesize testing};
    
    % horizontal and back up
        %arrow ending top right
    \coordinate(i10) at (21.15, 7.1); 
    \coordinate(i20) at (21.05, 7.0); 
    \coordinate(i30) at (21.15, 6.9); 
    
    \coordinate(i40) at (13.4, 6.9); 
    \coordinate(i50) at (13.4, 9.7); 
        % top   
    \coordinate(i60) at (13.3, 9.7); 
    \coordinate(i70) at (13.5, 9.9); 
    \coordinate(i80) at (13.7, 9.7); 
    
    \coordinate(i90) at (13.6, 9.7); 
    \coordinate(i100) at (13.6, 7.1); 
    \fill[scw-orange] (i10) -- (i20) -- (i30) -- (i40) -- (i50) -- (i60) -- (i70) -- (i80) -- (i90) -- (i100) -- cycle ;
    
    \node[scw-orange, left] at (13.4,9.2) {\footnotesize Fix};
    
    
    
    
    
\end{tikzpicture}
    \includegraphics[page=23, width=\textwidth]{03-education/figures/tikzfigures.pdf}
  \caption[OWASP Top 10 2021 draft]{Some categories from the \gls{owasp} Top 10 2017 decrease in priority (marked in orange) in the new draft, other categories increase in priority (marked in blue). Three categories are merged with other categories (marked in gray). In the 2021 draft, three new categories are introduced (marked in blue).}
  \label{fig:newowasptop10} 
\end{sidewaysfigure}


We see a clear gap between knowledge and practice with these vulnerabilities being so prevalent in practice, but relatively easy to fix in training.
This is because, in a custom setting such as the \gls{scw} training platform, the developer is aware of security and able to apply their knowledge to the examples at hand.
In practice, however, the developer is focused on the functionality and other requirements of the code, and security is no longer a priority.
The cognitive burden to constantly keep track of both the functionality and the security of the code is evidently too large.
With the right processes and the right tools this burden can be alleviated, and the prevalence of these vulnerabilities reduced.
This is the goal of Part~\ref{p:tools} of this work.

%\todo[inline]{Relate findings to some other research about the security of programming languages}

\subsection{Recommendations}
In the experiments described in the previous chapter, several algorithms were tested and adapted to learning systems.
Memory-based algorithms based on baselines have come out on top.
The best performing is the \gls{knn} baseline algorithm using the baseline-centered Pearson similarity measure.
This was expected based on the current exercise selection, as many users show a consistent bias in their ratings.
Model-based algorithms based on dimensionality reduction through matrix factorization, are often the best performing algorithms.
However, the advantages of these algorithms are often not applicable to our current data set.

\paragraph{Data sparsity}
In many recommender systems, the user-item rating matrix is rather sparse.
In Netflix, for example, few users will have watched even half the catalogue of movies.
This is also the case for the \gls{scw} training platform, especially for the largest frameworks that offer over a thousand challenges and by adapting the algorithms to learning systems, the sparsity of the data has even been increased artificially.
This data sparsity can cause some problems for \gls{cf} algorithms.

The \emph{cold start problem} occurs when a new user or item enters the system.
Because there is no item available about this user or item, it is difficult to find neighbouring items.
Matrix factorization techniques reduce the dimensions of the matrix to alleviate this problem.
In our data set only users have been included that completed a sufficient amount of challenges so that their ability level could be estimated, so this problem has been avoided.
In practice, it will still be necessary to calibrate the ability level of new users before accurate recommendations can be made.
One problem to avoid the cold start problem can then be to use a short entry test, in the form of \gls{cat}.
A procedure like this is present in other learning systems, such as Duolingo.

For new items, a trade-off needs to be made between exploitation and exploration.
In the exploitation phase the predictions from the algorithms are used to provide a recommendation to the user.
In the exploration phase, an item is recommended for which there is insufficient data, risking a bad recommendation in order to gather new information about this item.

\paragraph{Scalability}
When the number of users and items is excessively large, computing the similarity between every two users becomes an expensive procedure.
Model-based algorithms often scale better with large data sets because matrix factorization techniques are used for dimensionality reduction.
While there are many users on the \gls{scw} training platform, and the number of users is only expected to grow, in practice the data sets are split per framework.
They are not nearly large enough for scalability to be a problem, as similarity matrices for the \gls{knn} algorithms are computed in less than a minute.
These matrices only need to be computed once in a while, for example once or twice a day.

\paragraph{Synonyms}
Synonyms occur when a number of identical or very similar items have different names or entries in the data set.
Model-based techniques are capable of dealing with the synonym problem because they do not use the item names directly, but instead look for latent factors related to the items.
In our data set we do not expect many natural synonyms to exist.
While there are duplicate exercises, in the sense that they are in the same framework and about the same vulnerability type, in reality they are in different codebases, and of varying complexity.

With the learning adaptation, however, we have intentionally introduced synonyms by splitting items into separate entities based on the ability level of the users answering them.
The fact that we still see significant improvements in the model-based algorithms, who are supposed to factor out item names, is proof that these items demonstrate significantly different characteristics in the latent factor space.
This validates the hypothesis that user ability is an important factor for the recommendation of items in a learning system.
It is possible that user ability is represented in the latent factor space in one way or another.

\paragraph{Shilling attacks}
In some recommendation systems (such as for example the Amazon web store) users can be compelled to give positive recommendations towards their own material and negative recommendations towards their competitors.
While there is less incentive to do this type of intentional rating on the \gls{scw} training platform, similar scenarios have been detected.
Users in one company made it a competition to see who could gather the most points, and they created bots for this purpose.
The bots would randomly guess at first, and keep track of the correct answer for future attempts.
This resulted in several users who answered all exercises in a single framework several times over, causing worse ratings for those items as these users did not learn anything new according to the \gls{irt} estimates.
This has now been discouraged by preventing the same user from earning points through an exercise they have already solved in the past.
For the experiments of this work, data generated by these bots has been filtered out.

\paragraph{Implicit data}
Implicit data has been briefly introduced in the explanation of the SVD++ algorithm.
This algorithm not only characterizes users based on their ratings, but also based on which items they have rated.
Using implicit data like this is expected to have a big impact on prediction accuracy for systems where the user can choose items themselves.
In Netflix, for example, it can become apparent that a user constantly avoids watching movies of a certain genre, or that star a certain actor.
On the \gls{scw} platform we also see an improvement, most likely because this allows the algorithm to better distinguish between users who follow the recommended courses, and those who do not.

\subsection{Adaptation}
The proposed adaptations to learning systems in this work is not specific to software security and could be applied to other domains.
The adaptation based on processing of the data is especially easy to implement and can be applied to any \gls{cf} algorithm.
The biggest requirement is that sufficient data needs to be available to overcome problems caused by data scarcity which can be exaggerated by splitting the data even more.
In learning systems more so than in movie or music recommendations, we can expect users to rate a significant portion of the items, which makes this requirement more likely to be met.

The adaptation was less effective in model-based \gls{cf} algorithms.
One possible explanation is that the latent factors from the dimensionality reduction techniques already represented an ability estimate.
However, estimating this through the ratings alone is likely less accurate, which is why adding it more explicitly as a filter still improved the accuracy of the predictions.
It is possible to imagine a similar adaptation in other contexts where the latent factors might be doing a good job already, but small improvements can be made by computing an important factor explicitly.

%https://reader.elsevier.com/reader/sd/pii/S0950705109000161?token=5D449CD4740BAB0E3B98D60E31098B45CF1D75C3D69F058A1C397FA12A8BEB33E63692FE48A440A3190C8AE1739E8288&originRegion=eu-west-1&originCreation=20210930091431
%https://reader.elsevier.com/reader/sd/pii/S036013150400034X?token=E19FAC8FDAF7E1644408308AD34BC37A0A8657BEE89213ACC3400CB78EF55F59EC4DF5626C77FF12D4BE4262FDD3D103&originRegion=eu-west-1&originCreation=20210930092126
%https://citeseerx.ist.psu.edu/viewdoc/download?doi=10.1.1.661.3840&rep=rep1&type=pdf
%https://www.sciencedirect.com/science/article/pii/S036013150400034X
%https://www.pluralsight.com/content/dam/pluralsight2/product/iris/AdaptiveAssessments_af_v1.pdf
%https://repository.uel.ac.uk/download/e3b705a3e6417837ad1377af2f0fe65ee01cf91ecec3bda7a2a5257fca3d9808/129643/2011_Yarandi_etal_Item_Response_Theory.pdf
\section{Perspectives}
\label{sec:its-perspectives}

\subsection{Implementation into the training platform}
Results from the \gls{2pl} model can be used to improve the \gls{scw} training platform.
This is a step by step process that has already started.

\paragraph{Quality control of the exercises}
The results of a \gls{2pl} model for the use in tests are used to remove items with a low discrimination parameter from the item bank. 
A low discrimination parameter means that this item it cannot differentiate well between users of high and low ability levels.
Items like this are not useful in a test, where discriminating between users of different ability levels is exactly the goal.
Before removing items with a low discriminative ability, the examiner often manually checks items that are only slightly below the predetermined threshold.
If they are deemed important enough to be included in the tests despite their low discriminative ability, they are not removed from the item bank.

As a first step to use the results of the \gls{2pl} models in the \gls{scw} platform, we can use a similar procedure for quality control of the challenges.
In contrast with tests, estimating ability is not the main goal of the \gls{its}.
Items with a low discrimination parameter might have little influence on the accuracy of ability estimates, but these items could still provide meaningful learning opportunities.
A low discriminative ability can be explained by an extreme difficulty, or by a defect.
If all users are answering the challenge correctly because it is extremely easy, or all users are answering it incorrectly because it is extremely hard, then this challenge can not be used to discriminate between users of high and low ability level.
But if the discrimination parameter is low, and the difficulty level is not extreme, that means there is a different reason for this low correlation between a correct answer and the ability level of the user.
A low discrimination parameter, in this case, is an indication that the challenge is misleading or defect.
Challenges like this have been manually checked, and many of them have indeed shown defects in the past, or are still misleading in some way.

\paragraph{Improved challenge difficulty measure}
As explained in Appendix~\ref{app:challenges}, currently the difficulty of the challenges is only determined by the number of options to choose from.
It is a probability of answering correctly in case of a blind guess.
The results of the \gls{2pl} model experiment in Section~\ref{sec:eval-rasch} show that there is no statistically significant correlation between this difficulty and the probability of users answering the challenge correctly.
It is not an accurate difficulty measure.

Nonetheless, this difficulty measure is used in tournaments and other gamification features to decide the amount of points that are awarded to a user after answering correctly.
It would be more accurate to use the \gls{irt} difficulty for this purpose.
This difficulty is only available for challenges for which there is a sufficient amount of data.
There is still need for another measure to better approximate the real difficulty of new challenges that still have insufficient playtime.

In the experiment, I have shown that there is a correlation between the difficulty on one hand and the framework, the vulnerability type, and the presentation on the other hand.
A good start for such a temporary approximation could then be to take the mean difficulty of all challenges in the same framework, about the same vulnerability, and with the same presentation.

\paragraph{Improved user ability measure}
Currently, a security maturity score is computed for each user based on the amount of points they have earned and the accuracy they have maintained while doing so.
This maturity score is shown on the metrics dashboard, together with a more granular breakdown of the average strengths and weaknesses.
This metrics dashboard is shown in Figure~\ref{fig:metrics}, on page \pageref{fig:metrics}.
It is easy to reach a high maturity level for any user, if they spend enough time solving many easy challenges so that they earn points while maintaining a high accuracy.

The \gls{irt} ability estimate can be used to make this maturity score more meaningful.
However, it currently only provides a global ability estimate and lacks the granularity required for the metrics dashboard.
A multidimensional \gls{irt} model can be used to achieve this.
It remains future work to train and evaluate such a multidimensional \gls{irt} model with, for example, one dimension for each vulnerability category on the platform.

\paragraph{Improved assessments}
The past few months, assessments have been the most used play mode on the platform.
Assessments are built like classic tests, all users have to complete the same challenges and their accuracy is used as an indication of ability.

First, assessments can be improved by using \gls{irt} to estimate the ability level of the user.
This ability estimate is more meaningful than the accuracy.
This is most easily understood with an example: two users each complete an assessment with two challenges, one easy challenge and one difficult challenge.
The first user makes a mistake (due to inattention) on the easy challenge, but answers the difficult challenge correctly.
The second user answers the easy challenge correctly, but does not know the answer to the difficult challenge.
These users have achieved the same accuracy on the assessment.
However, intuitively, we would be more likely to attribute a higher ability level to the first user.
This is exactly what can be achieved when \gls{irt} is used.

Second, assessments can be made adaptive, similar to a \gls{cat}.
This means, serving new challenges based on the temporary ability estimate during the assessment.
Not only will this improve accuracy of the ability estimate, it will also reduce the amount of challenges needed to complete an assessment.

\paragraph{Recommendations in training}
Once all other implementation related to the \gls{2pl} model have been completed, the \gls{cf} algorithm can be integrated in the platform to dynamically recommend challenges to each user in the training mode.

\subsection{Integrating with developer tools}
\label{sec:its-integration}
\Gls{scw} is currently developing integrations for several developer and security tools such as Jira\footnote{\url{https://marketplace.atlassian.com/apps/1221320/}}, GitHub\footnote{\url{https://github.com/marketplace/secure-code-warrior-for-github}}, Fortify\footnote{\url{https://www.microfocus.com/en-us/fortify-integrations}}, and more\footnote{\url{https://help.securecodewarrior.com}}.
Additionally, there is also an integration for the \gls{ide}, discussed in great detail in Part~\ref{p:tools} of this work.

The current goal of these integrations is to provide contextual training to developers.
When a security vulnerability is detected through Fortify, or a ticket that involves a vulnerability is created in one of the bug tracking systems, the integration will insert links to training on the \gls{scw} platform about this specific vulnerability.
This training is highly relevant since it is directly related to the task at hand, i.e. remediating the vulnerability.

However, it is my opinion that they might hurt the productivity and usability of the developer.
I believe the developer does not want to make a context switch to complete training exercises, when they should be resolving the problem.
It would be more beneficial to let these integrations insert project-specific remediation guidance, as will be explained in Part~\ref{p:tools}.

Most of the integrations are still in development, and are only used by a select few customers.
In the future, when more data is available, it will be interesting to analyse which of these inserted links are clicked most frequently.
This data can give an indication for which vulnerabilities developers seek out training, and hence which vulnerabilities are likely more difficult to understand and solve in practice.
It would be interesting to compare these results to those of the \gls{2pl} model and priority lists such as the \gls{owasp} Top 10.

\paragraph{Adaptive Security Guidance}
Instead of forcing a developer to make a context switch to follow this contextual training, I have invented an alternative solution.
With this system, contextual training can be provided at a more appropriate time, that is, when the developer opens the training platform.
This invention has been patented by \gls{scw} as a ``Method and System for Adaptive Security Guidance"\footnote{\url{https://patents.google.com/patent/US20200211135A1/en}}.

In this invention, the data collected from the integrations feeds into the \gls{its} to result in even more relevant, and highly applicable recommendations.
For example, the \gls{ide} integration allows us to monitor code changes the developer makes and verify them against a set of rules to detect the mistakes they make in practice.
Integrations with security scanners such as Fortify in combination with code repository tools such as GitHub, also allow us to collect data about the detected vulnerabilities and who is responsible for those pieces of code.
At the same time, integrations with issue tracking systems such as Jira allow us to detect which tasks a developer is assigned to, and whether there are any security problems or security-critical features among them.
All of this data, in combination with the performance of users on the platform itself, enables us to make highly relevant recommendations.

Implementing such a system is most likely achieved by first selecting a list of relevant items to recommend and then choosing the most appropriate.
If a user is assigned a ticket on Jira regarding a \gls{sql} injection, for example, a challenge needs to be selected from the list of challenges about \gls{sql} injection in the relevant language and framework.
Out of this list, the challenge with the highest predicted rating can be recommended to the user.
Implementing and evaluating this system remains future work.

\subsection{Mobile application}
During my research at \gls{scw} I have been closely involved in the design of a mobile application called Secure Code Bootcamp\footnote{\url{https://www.securecodewarrior.com/products/secure-code-bootcamp}}.
Besides videos and texts explaining different vulnerability types, developers can also play challenges similar to those on the training platform.
The challenges for the mobile app can be generated from the same vulnerability data as those on the platform.

Instead of presenting it as an identify, locate, or fix exercise, a new presentation form has been developed that is more suitable for a small screen.
In these challenges the user has to review a code sample and either accept or reject it based on the security of the sample.
To accept or reject, the user can tap a button, or swipe to the left or right in a Tinder-style interaction with the app. 

The app is used by a few hundred students and developers but has not yet had as much use as the \gls{scw} platform itself.
In the future, analyzing the learning behaviour of the users on this app can provide further insights in the effect of different types of exercises, and the mobile context of the education.