\chapter{Conclusion}
\glsresetall

In this work, the focus is on improving the usability of the developer with more targeted, role-specific training and tools.
I propose a methodology to improve collaboration between developers and security professionals, called the paved path methodology.
In line with this methodology, I made improvements to both training and tools provided by \gls{scw}.

\summarybox{
In the first part of this book, I designed and implemented an \gls{its}.
First, the trained Rasch model from the field of \gls{irt} offered some useful insights in the mental model of the developer.
To speed up the slow estimation procedure for this model, I created an approximation method that remained sufficiently accurate long after initial calibration.
This approximation enables an accurate and easy-to-implement way to adapt existing \gls{cf} algorithms to learning systems.

In part~\ref{p:tools}, I evaluated the \gls{ide} plugin, called Sensei.
The results from the experiments show that customized rules and real-time remediation guidance result in high engagement from the developers using the tool.
To encourage this customization of rules, a \gls{yaml} based syntax and \gls{ui} were designed and evaluated.
Security professionals reported that, through this \gls{ui}, customization of rules is easier than with comparable tools.
}

\section{Intelligent Tutoring System}
The \gls{scw} training platform is usable and relevant to the needs of the developers.
It provides defense training, teaching developers to recognize insecure code patterns and how to fix them.
The training is available in a wide variety of programming languages and frameworks.
However, there is still a significant part of users that only follow a minimal amount of training.
It is likely that the pacing of the predetermined courses and tournaments does not fit their needs.
This is confirmed through surveys, where a portion of the users indicate they feel bored due to too much repetition or frustrated because the content is moving too fast.

In this work, I aimed to improve the efficiency of the training by adapting the learning pace to each individual user.
To do this, I designed an \gls{its} consisting of three algorithmic components.
These are exercise selection, user ability estimation, and exercise difficulty estimation.

The latter two are achieved through calibrating a two parameter logistic (2PL) Rasch model from the field of \gls{irt}.
With the results of this model, I have contributed insights into the mental model of a developer.
I have shown that both the vulnerability category and the language and framework have a medium-sized effect on the difficulty of an exercise.
Some key findings are that languages that require memory management result in increased difficulty, as well as the use of frameworks.
For the vulnerability category, results showed that the size of the related code fragments is a big indicator of the difficulty.
Design flaws that generally require fixing of larger code fragments, showed a higher difficulty on average.

Because the training of the Rasch model is slow, especially for large data sets, I developed an approximation procedure that outperformed several existing approximation methods from literature.
This method could be useful in many situations where full calibration procedures are not appropriate.

Finally, the recommendation system itself was developed and evaluated.
In the set of \gls{cf} algorithms that were tested, model-based algorithms were outperformed by simpler memory-based algorithms.
This is likely because the advantages of model-based algorithms such as improved capacity to deal with data sparsity, scalability, and synonyms are not very applicable to the data set of the \gls{scw} training platform.

One of the main contributions of this work to the state of the art, is the proposed adaptation of \gls{cf} algorithms to learning systems.
These simple and easy-to-implement adaptations are applicable to many, if not all, \gls{cf} systems and have shown a significant increase in prediction performance for all algorithms that have been tested.

Whether or not this recommendation system leads to increased engagement remains to be tested after the \gls{its} is implemented into the \gls{scw} platform.
This process will be gradually completed in several distinct steps.
In the future, it is also the intention to extend the \gls{its} so that it uses data from integrations with other developer tools to provide even more targeted training recommendations.
Finally, the proposed adaptations to learning systems could be tested on more \gls{cf} algorithms and more datasets. 

The results from the Rasch model, however, also indicate that a gap exists between knowledge and practice.
Some of the most notorious vulnerability types in practice, such as injection vulnerabilities and \gls{xss}, are shown to be relatively easy to find and fix in training.
This suggests that education alone is insufficient.
It is evidently too hard for a developer to keep track of the security of the code at all times while they are focused on the functionality instead.
With the right tools, such as Sensei, this burden can be alleviated and the developer can be timely reminded of the security implications of their work.

\section{Sensei}
There have been great advancements in the field of software security, as explained in Chapter~\ref{ch:related}.
Despite this, vulnerabilities still seem to be present in all types of software.
This is because advancements in development methodologies have also greatly sped up development, resulting in fast iterative releases.
This increased rate of change has made the job of security professionals more challenging.
On top of that, security professionals are understaffed and cannot adequately assist each of the developers to fix their code.

A shift left movement is ongoing to attempt to address this problem.
In this movement, security becomes a shared responsibility among everyone involved in the \gls{sdlc}.
To help developers secure their own code, they are given security training and handed security tools.
However, security training and tools are often designed with security professionals in mind.
They are using a reactive approach, and scan for vulnerabilities.
This approach requires sufficient code and calling context to be completed, which means that developers often have to go back to the code, potentially long after it was initially developed.
This clearly does not integrate well in developer workflows, and it is no surprise that developers dislike and often disable these tools.
Instead, they should be handed role-specific tools, so that they can receive proactive guidance that helps them more effectively and efficiently produce secure code.

That is the intent of Sensei.
In the related work, we have seen a distinct lack of such security tools designed with the developer in mind.
With this work, we have created and evaluated such a tool.
I helped with the design and requirements of this tool that is implemented by the engineering team at \gls{scw}.
With Sensei, we take a fundamentally different approach, enforcing guidelines regardless of the context, as the code is being written.
This will help developers produce secure code from the start.

The contributions of this work include an evaluation of this tool and some of its specific features.
In the first experiment, I have evaluated the effect of highly customized rules and remediation guidance in the \gls{ide}.
I was able to compare the results of this experiment to those of similar experiments with other tools.
I found that both the customization of rules as well as the quick-fixes in the \gls{ide} itself lead to increased engagement and trust from the developers.

Sensei is not the only tool that offers an easy-to-read rule syntax, or features to more easily develop and test rules.
However, usability tests indicate that a \gls{ui} that helps the rule-writer create rules without requiring intimate knowledge of the rule syntax could be an important feature.
We are not currently aware of any other tools that provide such a \gls{ui}.
Interviews with security professionals indicate that features in this \gls{ui}, such as the live preview, are useful and make development of rules for Sensei easier than for comparable tools.
They also indicated that the requirement to create a quick-fix for a rule promotes closer cooperation with colleagues that have a development background.

Based on the results of this work and those of related research, I recommend businesses to consider developer usability more closely and to offer different role-specific tools to developers and security professionals in their teams.

Future research could further expand on the security outcome of a tool like Sensei.
Previous work indicates that adhering to secure coding guidelines and best practices leads to more secure code~\cite{lipfordimpact,votipka2020understanding}.
Other work has also shown that more vulnerabilities are eliminated if feedback cycles are shorter~\cite{sampaio2016exploring}.
The contributions of this work indicate that when highly applicable remediation guidance is available, developers actively make use of it.
In the future, the combination of these findings could be validated to confirm that real-time guidance and quick-fixes lead to more secure code.

\section{Paved path methodology}
New technology on its own will not turn the tide, but the proposed solutions in this work will make it easier to make the required shift in culture.
A shift towards more human-centered and empathy-driven software development processes and workflows.

This work started with the introduction of such a process, called the paved path methodology.
In his methodology, the security team should gradually build a paved path for developers to follow.
Together with the developers, they should build standards and patterns that guide the rest of the development team to develop secure code from the start.
The guidelines should be specific, \gls{api}-level instructions, that can be easily understood by developers, and avoid security jargon.

By building a paved path, the security professionals can more easily provide a service to developers, instead of forcing security testing on them.
A paved path approach like this can be encouraged with a tool like Sensei.
Because the security professional that is creating the rules is now forced to also create a fix.
To achieve this, they most likely need help from members of the development team or at the very least colleagues with a development background.
Interactions like this result in closer collaboration among the teams and hence more mutual empathy.

A shift like this does not happen overnight, and this work on its own might not radically change software development.
But with this method, I advocate to \textit{pave the path} towards secure development, and take a step in the right direction.
A step in the journey towards a more human-centered future of software security.