\chapter{Conclusion}


% Avoid exaggerating the applicability of your research. If you’re making recommendations for policy, business or other practical implementation, it’s generally best to frame them as suggestions rather than imperatives – the purpose of academic research is to inform, explain and explore, not to instruct.

%If you’re making recommendations for further research, be sure not to undermine your own work. Future studies might confirm, build on or enrich your conclusions, but they shouldn’t be required to complete them.

%Make sure your reader is left with a strong impression of what your research has contributed to knowledge in your field. Some strategies to achieve this include:
%Returning to your problem statement to explain how your research helps solve the problem.
%Referring back to the literature review and showing how you have addressed a gap in knowledge.
%Discussing how your findings confirm or challenge an existing theory or assumption.
%Again, here, try to avoid simply repeating what you’ve already covered in the discussion. Pick out the most important points and sum them up with a succinct overview that situates your project in its broader context.